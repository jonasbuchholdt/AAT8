\chapter{Final Conclusion}
In this chapter, a recapitulation of all results from the project work shall be given. This will especially take up on the questions that were stated in \autoref{sec:problem_statement}.\\
\paragraph{Measurement Routine}
A measurement routine, which allows determining the directional characteristics of sound sources such as a single loudspeaker or a loudspeaker array has been implemented (see \autoref{sec:sweep_theorie}). This has proven to be functional and useful. While the measurement routine itself has no direct connection to implementing directivity control, it is required to verify results.
\paragraph{Directional Characteristics of a Single Loudspeaker}
A single 13$^{\prime \prime}$  was characterized in \autoref{sec:beam_width} and an augmented analytical model has been developed in \autoref{sec:correction}. According to the measured data, omnidirectionality seems to be a fair assumption up to a frequency of approx. \SI{100}{\hertz}. At higher frequencies, this assumption still approximates the behaviour of the loudspeaker, but there are some noticable deviations. These manifest especially when there is more then one loudspeaker in the close proximity to another one. This has been measured in the context of \autoref{ssec:ideal_approach}. The concept of an acoustic center seems applicable to the whole frequency range of this project, but it is suspected that the position of the acoustic center shifts when instead of being in an undisturbed soundfield a loudspeaker is placed in the proximity to other loudspeakers.
\paragraph{Array Models}
Three models were utilized, based on the analytical description of omnidirectional sources, the formerly mentioned augmented model, and on \gls{fdtd} simulation. Their performance was assessed based on measurement results in \autoref{sec:meas_vs_theory}. It can be concluded, that all three models are capable of giving a solid qualitative estimate of the directional characteristics and each have particular advantages over the other models at certain frequencies. The comparison also showed, that the augmented model has a tendency to predict directional characteristics in a too ``optimistic'' manner. However, it has not been investigated how the augmented model behaves, when there is correction tables derived from the polar response measurement of the individual speakers in the array instead of the measurement of a single speaker. Doing so, might bear some improvements in terms of precision.


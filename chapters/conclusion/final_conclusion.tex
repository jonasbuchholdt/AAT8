\chapter{Final Conclusions}
In this chapter, a recapitulation of all results from the project work shall be given. This will especially take up on the questions that were stated in \autoref{sec:problem_statement}.\\
\paragraph{Measurement Routine}
A measurement routine, which allows determining the directional characteristics of sound sources such as a single loudspeaker or a loudspeaker array has been implemented (see \autoref{sec:sweep_theorie}). This has proven to be functional and useful. While the measurement routine itself has no direct connection to implementing directivity control, it is required to verify results.
\paragraph{Directional Characteristics of a Single Loudspeaker}
A single 13$^{\prime \prime}$  was characterized in \autoref{sec:beam_width} and an augmented analytical model has been developed in \autoref{sec:correction}. According to the measured data, omnidirectionality seems to be a fair assumption up to a frequency of approx. \SI{100}{\hertz}. At higher frequencies, this assumption still approximates the behaviour of the loudspeaker, but there are some noticable deviations. These manifest especially when there is more then one loudspeaker in the close proximity to another one. This has been measured in the context of \autoref{ssec:ideal_approach}. The concept of an acoustic center seems applicable to the whole frequency range of this project, but it is suspected that the position of the acoustic center shifts when instead of being in an undisturbed soundfield a loudspeaker is placed in the proximity to other loudspeakers.
\paragraph{Array Models}
Three models were utilized, based on the analytical description of omnidirectional sources, the formerly mentioned augmented model, and on \gls{fdtd} simulation. Their performance was assessed based on measurement results in \autoref{sec:meas_vs_theory}. It can be concluded, that all three models are capable of giving a solid qualitative estimate of the directional characteristics and each have particular advantages over the other models at certain frequencies. The comparison also showed, that the augmented model has a tendency to predict directional characteristics in a too ``optimistic'' manner. However, it has not been investigated how the augmented model behaves, when there is correction tables derived from the polar response measurement of the individual speakers in the array instead of the measurement of a single speaker. Doing so might bear some improvements in terms of precision.
\paragraph{Speaker positioning}
The positioning of the speakers has been initally been intended to be a part of the optimization algorithm but had to be fixed due to practical constraints regarding the practicability when building the array and keeping the beamforming cost to an acceptable extent. It could be observed, that the placement of the speakers has an influence on the effectiveness of the directivity control as well as on the main axis frequency response. For the sake of implementing an array and measuring its response with only one main lobe direction, the positions have been fixed to an isosceles triangle with the same basewidth and height. If an array with an arbitrary mainlobe direction in relation to the speaker positions should be implemented, an equilateral triangle might be better suited. For any implementation of a triangular array, the dimensions of the triangle are set in a trade-off between main axis frequency response, beamforming effectiveness and evenness when shifting the main lobe along the circumference.
\paragraph{Finding \gls{sp}-parameters}
In order to generate \gls{sp}-parameters to enable the speakers to beamform, a \gls{ga} has been set up and utilized (see \autoref{sec:genetic_implememtation}). Some simplifications have been exploited regarding the direction of the main lobe in relation to the speakers in order to reduce the effort that has to be undertaken to implement those parameters. The \gls{ga} has been a suitable tool for development, being flexible enough to account for changes in the speaker configuration and speaker modeling.\\
The term ``beamforming cost'' has been introduced in order to describe the main axis frequency response of the triangular array. This cost can be calculated analytically (see \autoref{ssec:main_axis_formulas}) and is used to specify a cost filter, that equalizes the main axis frequency response.
\paragraph{Implementing \gls{sp}-parameters}
The required \gls{sp}-parameters have been implemented offline as filters in \autoref{sec:filter_design}. The filter, that enables the speaker array to beamform is a \gls{fir} filter, which is derived from the results of the formerly mentioned \gls{ga}, making use of yet another \gls{ga}. The cost filter is set up as an \gls{iir} filter. Both have been implemented offline with sufficient accuracy in order to achieve the intended behaviour of the array, as it has been confirmed by the measurement of the directional characteristics of the array (see \autoref{sec:meas_vs_theory},\autoref{sec:beamforming_array_spl}). 
\paragraph{Suitability of \gls{fdtd}-modelling}
\gls{fdtd}-simulation has been successfully implemented and utilized to model the behaviour of the loudspeaker array in free field conditions and derive estimates of the directional characteristics (see \autoref{the_simulation_result}) as well as within rooms (see \autoref{sec:dis:simulation}). It has been shown, that the predicting the directional characterstics of the speaker array based on \gls{fdtd} leads to reasonable results. The simulation results for rooms show, that beamforming also has noticable effects on the sound field in non-free-field conditions. These simulations have not been confirmed by any measurements. The most critical part when simulating rooms appears to be finding suitable parameters for the boundaries.
\paragraph{Prospects}
An overview over some possible applications for the array technology has been given in \autoref{sec:applications}. Some more information about the behaviour of the speakers and the array could be derived from doing measurements according to the ``ideal'' approach that has been sketched in \autoref{ssec:ideal_approach}. Also, the influence of the size of the loudspeakers and the shape of their cabinets is a subject might be worth investigating. There have been no measurements investigating, how the array behaves with the mainlobes being beamformed in different directions in relation to loudspeakers. An investigation on how the lobes are influenced by the non-omnidirectional behaviour of the speakers at higher frequencies when the main lobe is not formed in the direction of the main axis of the single speakers in that make up the array might be of interest. Additionally systematic comparison of \gls{fdtd}-simulations and measurements under non-free-field conditions could bear insights.
Finally, the idea of a triangular speaker array could be adapted to three dimensional beamforming by adding at least one more speaker out of plane to the array. While introducing complication this might open up a whole new range of applications.\\
In principle, all of the investigated principles should be applicable to even lower-frequency sound. Extending the frequency range towards higher frequencies might prove difficult because of the lack of omnidirectional sound emission inherent to commonly used loudspeakers. By extending the frequency to a lower range also introduce that the pressure cost gets higher in the low frequency. Therefore if the beamforming shall be extended to the low frequency, another sub frequency speaker array might benefit, such that the non matching phase do not kills the low end frequency in the front of the array.

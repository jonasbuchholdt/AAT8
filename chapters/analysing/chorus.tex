\subsection{Chorus and Flanger Effect} \label{sec:chorus} 

%\label{chor_flang}

The chorus effect takes a single audio signal as input and applies different delay values to it \citep{chorus_gibson} \citep{chorus_apple}. Chorus effect using one delay is called flanger. Each of the delayed signals are then mixed with the original audio input. \\
A \gls{lfo} can be used to make the delay times vary. Different \gls{lfo}s can be used for each of the delay channels to make a richer sound mix and avoiding repetitive sound, but it implies more computations. The same \gls{lfo} can be used for all the delay channels but not at the same cycle for each delay \citep{chorus_testtone}. \\ 

Different parameters on the chorus effect. Some of these are:\\
\begin{itemize}
\item \textbf{Delay time}: The time difference between the original sound and the delayed one (The frequency of the signal from the \gls{lfo}).
\item \textbf{Chorus size}: The number of delayed sounds that will be mixed.
\item \textbf{Depth}: The amplitude of the signal from the \gls{lfo}.
\item \textbf{Waveform}: The waveform of the signal from the \gls{lfo} can be changed to triangle, sine, log ect. \citep{hobby_hour_chorus}
\item \textbf{Gain}: The amplification of the delayed signal.
\end{itemize} \citep{chorus_parameters}

A block diagram for  the chorus and flanger effect is shown in \autoref{fig:chorus_diag}.

\begin{figure} [htbp!]
	\centering
\begin{picture}(0,0)%
\includegraphics{chorus_diag.pdf}%
\end{picture}%
\setlength{\unitlength}{4144sp}%
%
\begingroup\makeatletter\ifx\SetFigFont\undefined%
\gdef\SetFigFont#1#2#3#4#5{%
	\reset@font\fontsize{#1}{#2pt}%
	\fontfamily{#3}\fontseries{#4}\fontshape{#5}%
	\selectfont}%
\fi\endgroup%
\begin{picture}(6327,3018)(3766,-3493)
\put(6706,-1366){\color[rgb]{0,0,0}LFO}%

\put(8101,-1636){\color[rgb]{0,0,0}Gain}%

\put(3781,-646){\color[rgb]{0,0,0}Input}%

\put(9406,-646){\color[rgb]{0,0,0}Output}%

\put(6571,-1906){\color[rgb]{0,0,0}Delay}%

\put(6616,-3211){\color[rgb]{1,0,0}Delay}%

\put(8101,-2986){\color[rgb]{1,0,0}Gain}%

\put(6706,-2671){\color[rgb]{1,0,0}LFO}%

\end{picture}%



\caption{Block Diagram of the chorus and the flanger effect.}
\label{fig:chorus_diag}
\end{figure}


As it can be seen on the block diagram in \autoref{fig:chorus_diag}, a signal that has not been affected by any changes is added to the same signal delayed, controlled by the \gls{lfo}. The addition is done just before the output. The chorus effect is represented by the black and the red parts of the block diagram. The flanger effect is represented only by the black parts of the block diagram. In \autoref{fig:chorus_and_flanger_time} the impulse responses of the chorus and the flanger effect are shown. It is seen that with a pulse as the input, in the flanger effect, the output will be the original pulse, followed by an echo. The period before this echo arrives varies, based on the \gls{lfo}. When using the chorus effect several echoes will arrive, still with varying delay time. 

\begin{figure}[htbp!]
\centering
\def\svgwidth{\columnwidth}
\scalebox{0.8}{\input{figures/analysing/chorus_and_flanger_time_domain.pdf_tex}}
\caption{Impulse response of the flanger effect (black) and the chorus effect (red and black).}
		\label{fig:chorus_and_flanger_time}
\end{figure}











\section{Conclusion of directional characteristic}
Several measurements (see Appendices \ref{ax:directional_1} \& \ref{ax:directional_2}) were conducted on the \gls{dut}, consisting of a \citep{seas33} inside a (400x400x400)\si{\milli\meter} cabinet. The loudspeaker proved to be suitable for sound playback in the frequency range of \SIrange{60}{300}{\hertz}, that this project focuses on.\\
The directivity within this frequency range is reasonably close to omnidirectional. With a threshold of \SI{-1}{\decibel}, the loudspeaker can be regarded as an omnidirectional source up to approx. \SI{100}{\hertz}. With a higher threshold of \SI{-3}{\decibel} this upper frequency limit is approx. \SI{200}{\hertz}. By implication this means, that the behaviour of the speaker cannot be described as omnidirectional above the formerly mentioned frequencies.
For the grand scheme of things this means, that it is likely possible to implement the concept of three-speaker-beamforming with arbitrary main lobe direction in the lower part of the selected frequency range. However in the upper part of the frequency range, adaptions to the non-omnidirectional characteristics of the speaker might have to be implemented in order to achieve satisfactory results.
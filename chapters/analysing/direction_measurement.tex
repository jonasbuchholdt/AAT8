\section{Origin}
Because this project is about shaping the directional characteristics of loudspeakers arrangements towards a particular direction, it is important to thorougly investigate the directional characteristic of a single loudspeaker cabinet. This serves as a baseline for comparison and also is essential, because the investigated cabinet will be used to form the speaker array later on.\\
In general, loudspeakers tend to display different directional behaviour depending on the frequency emitted. At low frequencies they can be viewed as omnidirectional sound sources. At higher frequencies the main direction of sound emission is in line with the motion direction of the voice coil. \citep{crocker98}
Depending on the ratio of the emitted wavelength to the diameter of the speaker, a radiation pattern with side lobes can occur. An analytic approximation to the behaviour can be made  when looking at a vibrating piston in an infinite baffel. However, this only takes into account the front side of the speaker. It is difficult to incorporate the effects of an enclosure into this model.\\
There are possibilities to numerically model the sound field around a speaker in a cabinet. However in the context of this project, conducting a measurement seems to be the most favourable approach towards quantifying the sound emitted by loudspeaker in the cabinet at numerous frequencies.
The knowledge gained through the measurements can then be used in order to designate a feasible frequency range for the beamforming.
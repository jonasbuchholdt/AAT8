\chapter{Test of the Guitar's Output Impedance}\label{app:output_impedance}
A test was made to get an overview of the output impedance of a guitar.

\section*{Materials and Setup}
To measure the output impedance on a guitar, the following materials are used:
\begin{itemize}
\item Digilent Analog Discovery 2 (Oscilloscope)
\item Fender Squier Classic Vibe Telecaster (Guitar)
\item Digilent Waveforms 2015 (PC - software)
\end{itemize}

\begin{figure}[htbp!]
\centering
\begin{picture}(0,0)%
\includegraphics{guitar_output_impedance.pdf}%
\end{picture}%
\setlength{\unitlength}{4144sp}%
%
\begingroup\makeatletter\ifx\SetFigFont\undefined%
\gdef\SetFigFont#1#2#3#4#5{%
  \reset@font\fontsize{#1}{#2pt}%
  \fontfamily{#3}\fontseries{#4}\fontshape{#5}%
  \selectfont}%
\fi\endgroup%
\begin{picture}(4205,2045)(4129,-2773)
\put(7561,-871){$-$}%
\put(6346,-1996){$Osc$}%
\put(6346,-2176){$Ch2$}%
\put(7156,-916){$Osc$}%
\put(7156,-1096){$Ch1$}%
\put(7966,-1996){$Osc$}%
\put(8011,-2176){$W1$}%
\put(7066,-1591){$R$}%
\put(4681,-2086){$Guitar$}%
\put(6346,-1771){$+$}%
\put(6346,-2401){$-$}%
\put(6931,-871){$+$}%
\end{picture}%
\caption{Setup for measuring the length of output impedance on a guitar.}
		\label{fig:appendix:guitar_output_impedance}
\end{figure}

\begin{figure}[htbp!]
\centering
\begin{picture}(0,0)%
\includegraphics{guitar_output_phase_impedance.pdf}%
\end{picture}%
\setlength{\unitlength}{4144sp}%
%
\begingroup\makeatletter\ifx\SetFigFont\undefined%
\gdef\SetFigFont#1#2#3#4#5{%
  \reset@font\fontsize{#1}{#2pt}%
  \fontfamily{#3}\fontseries{#4}\fontshape{#5}%
  \selectfont}%
\fi\endgroup%
\begin{picture}(4205,1464)(4129,-2773)
\put(7426,-1996){Osc}%
\put(6346,-1996){Osc}%
\put(6346,-2176){Ch2}%
\put(7966,-1996){Osc}%
\put(8011,-2176){W1}%
\put(4681,-2086){Guitar}%
\put(6346,-1771){+}%
\put(6346,-2401){-}%
\put(7426,-1771){+}%
\put(7471,-2401){-}%
\put(6796,-1591){R}%
\put(7426,-2176){ch1}%
\end{picture}%
\caption{Setup for measuring phase of output impedance on a guitar.}
		\label{fig:appendix:guitar_output_impedance_phase}
\end{figure}


\newpage

\section*{Test Procedure}


\begin{enumerate}
\item The materials are set up as in \autoref{fig:appendix:guitar_output_impedance}.
\item The Digilent Waveform 2015 is set as a Network analyser.
\item  The guitar is set to use the neck pickup and the volume and tone control are tuned all the way up.
\item  The network analyser is set to measure $V_{peak}$ from \SI{10}{\hertz} to \SI{22}{\kilo\hertz} with 5000 samples and the resistor R is chosen to \SI{10}{\kilo\ohm}.
\item The measured data on the Osc ch1 and ch2 is used in the following formula $\left | Z_o \right | = \frac{V_{Ch1}}{V_{Ch2}}\cdot R$ which depends on the frequency. 
\item The same procedure is done on the bridge pickup, and with both neck and bridge pickup activated.
\item Next the materials are set up as in \autoref{fig:appendix:guitar_output_impedance_phase}.
\item Step 2 is repeated and the  phase difference between the signal on channel 1 and the signal on channel 2, is measured from \SI{10}{\hertz} to \SI{22}{\kilo\hertz} with 5000 samples and the resistor R chosen to \SI{10}{\kilo\ohm}.
\item All data is calculated from polar to rectangular form by the formula $Z=\left | Z_o \right | \cdot (cos(phase) + j \cdot sin(phase))$.
\item The data is plotted in MATLAB.
\end{enumerate}

\newpage
\section*{Results}

\begin{figure}[htbp!]
	\centering
		\includegraphics[width=1\textwidth]{real_impedance.pdf}
		\caption{Measurement of the real output impedance of the three pickup settings.}
		\label{fig:appendix:real_impedance}
		\includegraphics[width=1\textwidth]{imag_impedance.pdf}
		\caption{Measurement of the imaginary output impedance of the three pickup settings.}
		\label{fig:appendix:imaginary_impedance}
\end{figure}

On \autoref{fig:appendix:real_impedance} it is seen that the lowest output impedance of the neck pickup is at \SI{10}{\hertz} with \SI{6220}{\ohm}  and the highest output impedance is \SI{58,57}{\kilo\ohm} at \SI{5104}{\hertz}.

On  \autoref{fig:appendix:real_impedance} it is seen that the lowest output impedance of the bridge pickup is at \SI{10}{\hertz} with \SI{7863}{\ohm}  and the highest output impedance is \SI{73,08}{\kilo\ohm} at \SI{4230}{\hertz}.

On  \autoref{fig:appendix:real_impedance} it is seen that the lowest output impedance of both the neck and bridge together is at \SI{10}{\hertz} with \SI{3563}{\ohm}  and the highest output impedance is \SI{53,43}{\kilo\ohm} at \SI{5702}{\hertz}.
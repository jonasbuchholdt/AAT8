\chapter{Measurement of Directional Characteristics III}\label{ax:directional_3}
This appendix serves as a protocol to a series of measurements conducted between the 10\textsuperscript{th} and 12\textsuperscript{th} of May  2018 in the large anechoic chamber (B4-111) at the acoustic lab of Aalborg University at Fredrik Bajers Vej 7.\\
The goal of these measurements is investigating the behaviour of a loudspeaker array consisting of three of the loudspeakers that have been featured in \autoref{ax:directional_1} and \ref{ax:directional_2}.

\section*{Measuring Equipment and Materials}
The following measuring equipment was used:
\begin{itemize}[noitemsep]
\item Microphone \gls{bandk} 4144
\begin{itemize}[noitemsep]
\item AAU-number: 06552
\item Serial number: 297090
\end{itemize}
\item Preamplifier GRAS 26AK
\begin{itemize}[noitemsep]
\item AAU-number: 56526
\item Serial number: 32810
\end{itemize}
\item Power supply \gls{bandk} 2636
\begin{itemize}
\item AAU-number: 08022
\item Serial number: 
\end{itemize}
\item Calibrator \gls{bandk}\ 4231
\begin{itemize}[noitemsep]
\item AAU-number: 33691
\item Serial number: 2115338
\end{itemize}
\item 2 pcs. Power Amplifier Pioneer A-616
\begin{itemize}[noitemsep]
\item AAU-number: 08249, 08699
\item Serial number: HJ9404841S, JG9405804S
\end{itemize}
\item Sound card RME Fireface UCX
\begin{itemize}[noitemsep]
\item AAU-number: 108230
\item Serial number: 23811948
\end{itemize}
\item Turntable: Outline ET 250-3D
\begin{itemize}
\item Serial number: REIB0012
\end{itemize}
\item MATLAB r2017b on OSX 10.11.6
\item 3 pcs. Loudspeaker SEAS 33 F-WKA
\end{itemize}

The following material was used:
\begin{itemize}[noitemsep]
%\item \SI{1/2}{\inch} to \SI{1}{\inch} preamp adapter
\item Microphone clip
\item Microphone stand
\item LEMU cable
\item \gls{bandk} cable
\item XLR cables
\item Ethernet cable
\item miscellanious adapters
\item 3 pcs. Loudspeaker cabinet, plywood, outside dimensions: (400x400x400)\SI{}{\milli\meter}, wall~thickness:~\SI{20}{\milli\meter}, equipped with \citep{seas33}
\item Speaker mount for turntable
\begin{itemize}[noitemsep]
\item Steel mounting contraption, see REFERENCE TO HARDWARE CHAPTER
\item 3 speaker legs, \SI{40}{\milli\meter} box section, top: aluminium, {\(\varnothing\)~:~\SI{34.8}{\milli\meter}}, height: \SI{1}{\meter}
\item Circular \gls{mdf} cutout, thickness: \SI{12}{\milli\meter}, {\(\varnothing\)~:~\SI{800}{\milli\meter}}
\item \gls{mdf} cutout, thickness: \SI{20}{\milli\meter}, surface: approx. (300x600)\SI{}{\milli\meter}, bolt pattern drilled according to the bottom side of the ET 250-3D turntable
\item \gls{mdf} cutout for counterweight mounting, thickness: \SI{12}{\milli\meter}, surface : (400x400)\SI{}{\milli\meter}
\item triangular cutout (isosceles), thickness: \SI{12}{\milli\meter}, approximate outside dimensions (base width x height): (970x600)\SI{}{\milli\meter}
\item 2 Electrovoice S-200 speakers as counterweight
\item Ratchet strap
\item 4 bolts M8x80, associated washers
\item 6 bolts M8x30, associated washers
\item 8 sinkhead bolts M8x40, associated nuts and washers
\item miscellaneous woodscrews

\end{itemize}
\end{itemize}



\section*{Setup}
When setting up the speaker array according to dimensions that have been decided upon in \autoref{sec:opt_result}, a substential effort was undertaken to insure a secure stance. The turntable was mounted to one of the platform grids used in the anechoic chamber by screwing bolts through a \gls{mdf}-plate and the grid into the threads on the underside of the turntable. A round \gls{mdf} plate was mounted using the boltpattern on the upside of the turntable. Fixed to the round plate a custom made steel contraption (\autoref{fig:speakerstand}) was bolted, which itself held the legs of the speakers in place. while allowing some adjustability. In a height of approx \SI{1}{\meter} above the turntable the three speaker cabinets were mounted to the legs. On top of the speaker a triangular \gls{mdf} cutout was employed to enhance rigidity. A picture of the array is given in \autoref{fig:05_11_setup}.

\begin{figure}[h]\label{fig:05_11_setup}
	\centering
    \includegraphics[width=0.5\textwidth]{05_11_setup.jpg}
    \caption{Speaker array setup on the turntable in the anechoic chamber.}
\end{figure}

The height of the centers of the speaker cabinet above the grid turned out to be \SI{1.37}{\meter}. The microphone was set up in the same height. Platforms for the speaker array and the microphone were set up in opposing corners of the anechoic chamber, which resulted in a horizontal distance between the array center and the microphone of \SI{4.92}{\meter}.
The input and the output gain of the \gls{bandk} 2636 microphone power supply were both set to \SI{+10}{\decibel}. The input gain on the Fireface UCX was set to \SI{0}{\decibel}. On the playback side, the power amplifiers had a fixed voltage gain of \SI{+10}{\decibel} and the \texttt{playgain}-parameter in the measurement routine was set to \SI{-10}{\decibel}. The playback signals were normed so that the absolute of the biggest amplitude in on of the filtered sweeps (see \autoref{sec:filter_design} is a digital value of 1.\\
The desired positioning of the acoustic centers of the speakers was found to be describable as \texttt{Lx}$\,=\,\SI{40}{\centi\meter}$ and \texttt{Ly}$\,=\,\SI{-40}{\centi\meter}$ in \autoref{sec:opt_result}. In order to make achieve the positioning without the loudspeaker cabinets physically interfering, the azimuth of both of the two front loudspeakers \texttt{B} and \texttt{C} was rotated by \SI{50}{\degree} inwards. The speaker positions were adjusted according to the finding of \autoref{ax:directional_2}, that the acoustic center of the speaker is approx. \SI{17}{\centi\meter}. The array center was set on the rotational axis of the turntable, in order to match the way, the point sources are set up in the analytical model and the \gls{fdtd} simulation.
This positioning was used as a baseline for measurements and was later adjusted to account for imprecisions and assymetries in the way the speakers were set up and further tweaked with an heuristic approach to changing the positions in order to achieve the best possible result with the given hardware and without changing the \gls{sp}-parameters. It is assumed that these tweeks were necessary in order to account for the influence that the speaker cabinets had on the sound field. By the time of the final measurements, the rear speaker \texttt{A} had been moved towards the front by \SI{4.5}{\centi\meter}, the side speakers had been moved towards the inside by \SI{2}{\centi\meter} each, and their azimuth angle had been increased to approx. \SI{55}{\degree}.
To illustrate the way, the speakers were set up, a birdseye view is given in \autoref{fig:array_pic}.
\begin{figure}[h]\label{fig:array_pic}
	\centering
    \includegraphics[width=0.5\textwidth]{speaker_array.jpg}
    \caption{Speaker array arrangement}
\end{figure}

\section*{Results}
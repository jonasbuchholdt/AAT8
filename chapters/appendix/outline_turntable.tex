\chapter*{Outline ET 250-3D turntable control}\label{appendix:turntable}
In this appendix the control of an Outline ET 250-3D turntable will be described. The turntable can be controlled in three ways, first the turntable can be controlled by \gls{ttl} commands through a jack connector. Secondly the turntable can be controlled by specified \gls{dll} command through Ethernet. All commands is included in the software packed folder. Thirdly the turntable can be controlled by \gls{udp} command through Ethernet. For controlling the turntable by MATLAB, the two Ethernet based control method is the easiest to do because MATLAB support Ethernet access. The \gls{dll} method require a complicated scripted which might only work on Windows operation system. The \gls{udp} can run at all operation system which support IPv4 Ethernet connection and is a short simple script, where the script open a \gls{udp} channel as a file, and e.g.  the script shall only edit the file in the right position to move the turntable. The chosen control method is \gls{udp} because it is simple and works at all modern computer operation systems.

\section*{Materials and setup}
The following materials are used:
\begin{itemize}
\item Outline ET 250-3D (Turntable)
\begin{itemize}[noitemsep]
\item AAU-number: -
\item Serial number: REIBO012
\end{itemize}
\item MATLAB 2017b (PC - software)
\item Ethernet cable
\end{itemize}



\section*{The \gls{udp} setup of the computer}
To establish connection between the turntable and computer, both have to run at the same SUBNET MASK. The turntable have a factory set for Ethernet connection which is as following:

\begin{table}[H]
\centering
\caption{Turntable network address}
\label{udp_setup_for_computer}
\begin{tabular}{lll}
 & \gls{ip} & 192.168.1.34   \\
 & SUBNET MASK  & 255.255.255.0   \\
 & DEFAULT GATEWAY  & 192.168.1.250  \\
 & BROADCAST IP   &  192.168.1.255   
\end{tabular}
\end{table}



\section*{Turntable control command}
The software is made as a function where one can get a position, specify a position and stop the turntable. The function name is:

\includeCode{ET250_3D.m}{matlab}{1}{1}{The turntable control function}{code:ET250_3D_function}{./code/turntable/}

The function is made as an switch case with input variable "cmd", and an angle input. The following command can be send to the "cmd" of the function:

 \begin{table}[H]

\caption{Function commands}
\label{udp_command}
\begin{tabular}{lll}
 & \texttt{cmd} = \color{Violet}{\texttt{'udp_start'}} & Which start a connection on port 7000 \\
 & \texttt{cmd} = \color{Violet}{\texttt{'set'}} & Which move the turntable to the specified angle    \\
 & \texttt{cmd} = \color{Violet}{\texttt{'get'}} & Which get the position of the turntable   \\
 & \texttt{cmd} = \color{Violet}{\texttt{'stop'}}  & Which stop the turntable from moving \\
 & \texttt{cmd} = \color{Violet}{\texttt{'udp_stop'}} & Which stop the connection on port 7000 
\end{tabular}
\end{table}




\section*{The MATLAB function}

\includeCode{ET250_3D.m}{matlab}{1}{55}{The turntable control function}{code:ET250_3D}{./code/turntable/}





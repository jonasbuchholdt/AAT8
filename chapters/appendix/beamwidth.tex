\chapter*{Test a guitars frequency area}\label{appendix:beamwidth}
In this appendix the beamwidth calculation software will be briefly explained. The gold of the software is to find the $-n$\si{\decibel} beamwidth of an arbitrary speaker, where the polar response is measured. 

\section*{Materials}
To calculate the $-n$\si{\decibel} beamwidth, the following materials are used:
\begin{itemize}
\item MATLAB 2017b (PC - software)
\end{itemize}


\section*{Calculation procedure}
The $-n$\si{\decibel} beamwidth calculation software is build as a search function. It compare the front measurement with a measurement from a given angle and find the first frequency where the \gls{spl} difference between front measurement and the measurement from the given angle is $n$\si{\decibel}. The following step have to be done to be able to calculate the $-n$\si{\decibel} beamwidth.

\begin{enumerate}
\item The materials are set up as in \autoref{fig:measurement_setup}.
\item The polar response is measured according to \autoref{appendix:measuring_manual}
\item The acoustical center of the speaker is founded according to \autoref{sec:ac_center}
\item  The speaker is moved such that the acoustic center stays in the same point. 
\item The polar response is measured again according to \autoref{appendix:measuring_manual}
\item All transfer function from the polar response is calculated.
\item $n$ is chosen and the  $-n$\si{\decibel} beamwidth calculation software is executed.
\end{enumerate}


\section*{The MATLAB function}
\includeCode{minus_ndB_beamwidth.m}{matlab}{1}{52}{The $-n$\si{\decibel} beamwidth search software}{code:minus_ndB_beamwidth}{./code/beamwidth/}

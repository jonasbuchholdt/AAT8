\newcommand{\projectAbstract}{
This project deals with low-mid frequency directivity control.
The directional characteristics of a single loudspeaker are measured and analysed. An analytical model of the loudspeaker is established, so that also a beamforming array can be modeled. In order to do this, essential investigations regarding the position of the acoustic center of the loudspeaker are carried out. After pointing out the properties of the commonly known first order gradient speaker, a three speaker array is designed. Three speakers are chosen to overcome the disadvantage of the first order gradient speaker not being able to freely control its main lobe direction. In order to predict the behaviour of the speaker array in a non-free-field environment, a numerical model is set up. This project only implements beamforming in the frontal direction of the array but by changing the filters the beamforming angle could be changed. A genetic algorithm is chosen to optimize gain and phase for the individual loudspeakers. A suitable positioning scheme for the speakers is established. The optimal signal processing parameters are implemented as filters. For the filter design, another genetic algorithm is employed. The measured directional characteristics of the array are compared with the analytical and numerical models and compared with the directional characteristics of a commercial product.

}

\newcommand{\projectSynopsis}{
Synopsis
}
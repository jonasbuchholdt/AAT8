\subsection{Refactoring}\label{ssec:refactoring}
Refactoring has partly been an element of the mini project, as there were different implementations (naive, vector-operation, parallel) to be made. 
In the context of a larger software project, refactoring could be done on a larger scale with different elements of the code. However, the whole concept of refactoring is in contrast to the classical waterfall approach. In the modified waterfall approach that is employed in this mini-project, a refactoring step is included, perhaps to retrofit some of the advantages of agile programming.
\subsection{Optimization}\label{ssec:optimization}
One of the tasks listed in the problem description for the mini-project involves implementing the algorithm in a way, that parallel processing is used. This can be done for \gls{fdtd}-simulation. However, because of the nature of the computational tasks, it takes a great deal of effort to make an implementation using parallel computing on the CPU, that consistently delivers better performance than a well designed implementation without the explicit use of parallel computing.
A further step to optimzation might be to implement the part of the calculations to compute on a GPU and gain performance by doing so. This is not within the scope of the mini-project at hand.

When simulating a sound field, every calculation is depending on the result of precious calculations and has to be executed for each time step in the right order in serial. The exception are the individual particle velocities for each time step. This means, that the particle velocity \texttt{x, y, z} and the boundary can be calculated in parallel. Another optimization possibility is going from double precision to single precision. A tests has be made to see if this leads to any relevant calculation error (see \autoref{ssec:evaluation}).
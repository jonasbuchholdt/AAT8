\subsection{Refactoring}\label{ssec:refactoring}
Refactoring has partly been an element of the mini project, as there were different implementations (naive, vector-operation, parallel) to be made. 
In the context of a larger software project, refactoring could be done on a larger scale with different elements of the code. However, the whole concept of refactoring is in contrast to the classical waterfall approach. In the modified waterfall approach that is employed in this mini-project, a refactoring step is included, perhaps to retrofit some of the advantages of agile programming.
\subsection{Optimization}\label{ssec:optimization}
One of the tasks listed in the problem description for the mini-project involves implementing the algorithm in a way, that parallel processing is used. This can be done for \gls{fdtd}-simulation. However, because of the nature of the computational tasks, it takes a great deal of effort to make an implementation using parallel computing on the CPU, that comes even remotely close to a well designed implementation without the explicit use of parallel computing.
However, it might be possible to implement the simulation on a GPU and gain performance by doing so. 
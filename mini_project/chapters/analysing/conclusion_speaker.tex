\section{Conclusion}
Several measurements (see Appendices \ref{ax:directional_1} \& \ref{ax:directional_2}) were conducted on the loudspeaker configuration, consisting of a \citep{seas33} inside a (400x400x400)\si{\milli\meter} cabinet. The loudspeaker proved to be suitable for sound playback in the frequency range of \SIrange{60}{300}{\hertz} that is the focus of this project.\\
The directivity within this frequency range is shown to be very unfocused. With a threshold of \SI{-1}{\decibel} the loudspeaker can even be regarded as an omnidirectional source up to approx \SI{100}{\hertz}. With a higher threshold of \SI{-3}{\decibel} this upper frequency limit is approx. \SI{200}{\hertz}. By implication this means, that the behaviour of the speaker cannot be described as omnidirectional above the formerly mentioned frequencies.
For the grand scheme of things this means, that it is likely possible to prove the concept of three-speaker-beamforming in the lower part of the selected frequency range. However in the higher part of the frequency range, adaptions to the non-omnidirectional characteristics of the speaker have to be implemented in order to achieve satisfactory results.
\chapter{Conclusion}
Within in the context of the mini-project at hand, a sound field simulation has been implemented using a \gls{fdtd}-based method. Investigations regarding precision and performance have been conducted. It has been established, that in an acoustical context, there is advantage in terms of precision in using single precision variables instead of double precision.\\
Using single precision also resulted in a significant performance benefit.\\
The chosen development plan (modified waterfall model) turned out to be a suitable choice for the given project. This might have been due to the clear target definition and the small number of participants, which made it easy to keep track of the development. For more complex software projects a more agile approach might be better suited.
The test method, that has been implemented, checks results from the simulation grid for acoustic feasibility. It can not rule out every possible error in the simulation routine, but it is a suitable test to make problems less likely to occur.
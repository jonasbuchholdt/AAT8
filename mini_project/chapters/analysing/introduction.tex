\section{\gls{fdtd} for soundfield simulation}
According to the project requirements for the mini-project, the students are supposed to choose a computational aspect related to their semester project. This, in case of the mini-project at hand, is \gls{fdtd}-simulation.\\
When in acoustics the behaviour of sound sources in a room is to be investigated, there are basically for different approaches. The simplest approach is considering only the amount of energy, that is brought into the room and how it is absorbed. The most prominent example is Sabine's formula that can be used to estimate the reverberation time $RT_{60}$. This very simplistic method only has limited uses. More elaborate methods are based on raytracing, but because of physical limitations they can only represent the higher part of the human hearing frequency range accurately. The very low end of the frequency range can be covered by modal models, that are based on the room geometry and which are not feasible to use towards the mid frequencies. \gls{fdtd}-simulation follows the fourth approach, which is based on simulating wave behaviour and is most suited to the mid frequency range. \citep{finiteproblems}\\
Because the subject of the main project in the current semester is \textit{low-mid frequency acoustical beamforming}, an \gls{fdtd}-simulation can serve as helpful tool in order to estimate the behaviour of a speaker array under different acoustical conditions. As with most simulations, the goal is, to be able to evaluate the performance of different configurations of loudspeakers and signal processing parameters relatively quickly and accurately. This can aid in the overall development process and keeps the number of real world measurements to a minimum, saving time and resources.\\
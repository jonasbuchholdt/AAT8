\section{Design Considerations} \label{ch:considerations}
Most of the considerations in the algorithm are already explained in \autoref{sec:fdtd}, where the over all design is explained. The following list provides the name of functions in the implementations that have been made. 

\begin{itemize}
\item The pressure formula is named \texttt{_p()} 
\item The particle velocity in x direction is named \texttt{_vx()}
\item The particle velocity in y direction is named \texttt{_vy()}
\item The particle velocity in z direction is named \texttt{_vz()}
\item The particle velocity at the left boundary in x direction is named \texttt{_vxlb()}
\item The particle velocity at the right boundary in x direction is named \texttt{_vxrb()}
\item The particle velocity at the top boundary in y direction is named \texttt{_vytb()}
\item The particle velocity at the bottom boundary in y direction is named \texttt{_vybb()}
\item The input is a sound pressure emitted by a transparent source, which is calculated for every iteration and injected at the center point of the grid. 
\item The output of the simulation is the \gls{rms} pressure of every point in grid after a predetermined number of samples.
\end{itemize}

\section{Data type}
Since the algorithm estimates pressure and particle velocity in space with respect to time, both pressure and particle velocity are stored in a 4 dimensional array where one of the dimensions is the time. Both pressure and particle velocity are set up as shared variables, which have expected values of one or less, if for example the transparent source injects pressure with an amplitude of one. The pressure of the source(s) may vary depending on the application. It has been chosen to store the array as shared floats. Because the array data for a typical room exceeds a size of \SI{50}{\mega\byte} and all calculations are executed in functions, data transfer between parts of the code is avoided. Data is shared in way, where the pointer name is the direction name (\texttt{x,y,z}).


\section{Test Plan}
The test intended test method is to compare the pressure drop of the simulation with an analytical model, which is explained in detail in \autoref{ssec:qmetrics}. According to this, the pressure will attenuate by approx. \SI{6}{\decibel} for every double of distance in far field. Since the \gls{fdtd} is an approximation of the far field scenario, an analytical model will utilized to set up a doctest, where the test compares the pressure drop from the analytical model to the simulation. To keep the computation time down, the distances chosen for the pressure drop calculations are one and two meter. The test will be run with both double precision and single precision variables.

\section{Profiling and benchmarking}
Most of the code has to run in series, but the particle velocity can be run in both parallel and serial, and therefore benchmarking will be focused on this part. To do the benchmarking of the particle velocity, the time, at which the particle velocity calculation starts, and the time, at which the particle velocity calculation has finished, will be recorded. These two times will be subtracted, resulting in the calculation time. All three implementations (naive, single core, multicore) will be tested. Details can be found in \autoref{ssec:evaluation}.




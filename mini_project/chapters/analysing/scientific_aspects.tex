\section{\gls{nsc}-aspects to be investigated}
Because the reason for doing the mini-project at hand is learning about \gls{nsc}, some aspects of the latter have to be approached. One rather obvious point is the performance of the algorithm, where relative changes in computation time between different implementations can be measured and discussed. When talking about performance, one thing, that comes to mind is parallelization. It is advantageous, when a solution to a computational problem can be computed in parallel. However, for \gls{fdtd} the fact, that the algorithm relies on information that is computed in previous iterations of a loop, makes parallelization rather difficult. Operations like the multiplication of large matrices can still be implemented as parallel computation, but it is not possible to run iterations of the calculation in the pressure grid in parallel. However, for particle velocity matrices, which are directional, it might be possible to parallelize computations and also for a three dimensional case, further parallelization might be possible.
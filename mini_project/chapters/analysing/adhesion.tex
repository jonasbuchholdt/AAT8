\chapter{Adhesion to the Software Development Plan}\label{ch:sdp}
\section{Recap: The Waterfall Model}\label{sec:recap_waterfall}
In the software development plan, that had to be submitted as an excercise, following the \gls{nsc} lecture on the $28^{th}$ of March 2018, it was stated, that a Waterfall-based development method was to be followed during the course of this mini-project.
According to \citep[p. 141]{preview} and based on \citep{royce70} a modified Waterfall approach can be described with the following points:\\
\begin{itemize}
\item Functionality
\item Algorithm
\item Documentation
\item Coding
\item Quality Metrics
\item Evaluation
\item Refactoring
\item Optimization
\end{itemize}
Because the requirements for Mini-project are clearly specified, a plan-driven development model like the modified Waterfall scheme is suited to the software project. Another aspect, that helps to ensure a smooth development, is that, due to the group size, only two developers are working on the code. This enables a form of communication and up-to-date-keeping, that would be much more complicated when a larger number of participants were involved.
\section{Development Log}\label{sec:dev_log}
In this section, brief statements to how the points from \autoref{sec:recap_waterfall} were addressed in the course of the mini-project will be made.
\subsection{Functionality}\label{ssec:functionality}
Because the \gls{fdtd}-simulation is part of another project, which deals with low-mid frequency acoustical beamforming, many of the functional requirements were dictated by the usage in the main project.
\chapter{Adhesion to the Software Development Plan}\label{ch:sdp}
\section{Recap: The Waterfall Model}\label{sec:recap_waterfall}
In the software development plan, that had to be submitted as an excercise, following the \gls{nsc} lecture on the $28^{th}$ of March 2018, it was stated, that a Waterfall-based development method was to be followed during the course of this mini-project.
According to \citep[p. 141]{preview} and based on \citep{royce70} a modified Waterfall approach can be described with the following points:\\
\begin{itemize}
\item Functionality
\item Algorithm
\item Documentation
\item Coding
\item Quality Metrics
\item Evaluation
\item Refactoring
\item Optimization
\end{itemize}
Because the requirements for Mini-project are clearly specified, a plan-driven development model like the modified Waterfall scheme is suited to the software project. Another aspect, that helps to ensure a smooth development, is that, due to the group size, only two developers are working on the code. This enables a form of communication and up-to-date-keeping, that would be much more complicated when a larger number of participants was involved.
\section{Development Log}\label{sec:dev_log}
In this section, brief statements to how the points from \autoref{sec:recap_waterfall} were addressed in the course of the mini-project will be made.
\subsection{Functionality}\label{ssec:functionality}
Because the \gls{fdtd}-simulation is part of another project, which deals with low-mid frequency acoustical beamforming, many of the functional requirements were dictated by the usage in the main project. The usage in the main project demands, that the simulation routine sets up a \gls{fdtd}-grid with a given room size and different boundary conditions. Furthermore, one ore several transparent sound sources have to be placed in the grid at given positions and the sound emitted by those sources has to be individually adjustable in amplitude and phase. It must be possible to export plots of the pressure field at a given point in time in order to illustrate the behaviour of arrangement of sound sources in the simulation. In the main project, also polar plots are extracted from the \gls{fdtd}-data, but this is not part of the implementation made in the mini-project.
\subsection{Algorithm}\label{ssec:algorithm}
An algorithm is needed that insures that the formerly stated functional requirements are achieved. While the theoretical basics are described in \autoref{sec:fdtd}, some more detailed explanations on a more technical level are given in \autoref{ch:implementation}. These go hand in hand with \autoref{ssec:coding}. While the theory of \gls{fdtd} provides well established fundamentals, of what mathematical construct should be implemented, there is still a number of choices, regarding what the implementation should look like considering the intended application. 
\subsection{Documentation}\label{ssec:documentation}
Specific requirements on the documentation were given in the mini-project task description. For documentation purposes the document at hand has been written and the code itself has been augmented with comments, based on the proposals given in \citep{pep8}.
When looking at the modified Waterfall model, the documentation step is a point, that has had a more iterative nature than other points. This is mainly due to the fact, that, within the development process, some difficulties in the implementation have led to more complex coherencies then initially anticipated. Keeping the documentation up to date means coming back to it basically after every change that is being made to the software.
\subsection{Coding}\label{ssec:coding}
In the coding phase, everything that had been planned previously has been put into code. However, there was an iterative component when the evaluation (\autoref{ssec:evaluation}) revealed some faulty behaviour of the code. Coding has been executed by two different programmers who kept track of the versions via the means of \textit{git}. All code was written in Python 3.6.5. Several implementations of the code have been made, in order to investigate their behaviour in terms of efficiency (see also \autoref{ssec:optimization}).
\subsection{Quality Metrics}\label{ssec:qmetrics}
When quality metrics are to be considered, there are three main issues that come into mind. The functional requirements (\autoref{ssec:functionality}) have to be fulfilled, for reference cases the results should match those known to those cases that are obtained from trusted sources and finally the computation time should be as low as possible.\\
In order to design a routine that checks the feasibility of simulation results by comparing it to results, that are known to be correct, a suitable example of such a reference case has to be found. Independent of the boundary setup, the conditions in a finite size \gls{fdtd}-grid correspond to acoustical free field conditianons as long as the waves emitted by sources have not reached the boundary. This means, a correct behaviour of the simulation can be confirmed by investigating one of these ``early'' samples. Based on \citep[p. 172]{kinsler2000}, the sound pressure difference at two points with distances $r_1$ and $r_2$ from a pulsating sphere can be described as follows:
\begin{equation}\label{eq:deltalplong}
\Delta L_{p,a}(r_1,r_2)\,=\,20\log_{10}\frac{|j\rho_0 c V_0(\frac{a}{r_1})ka \textbf(e)^{j(\omega t-kr_1)}|}{|j\rho_0 c V_0(\frac{a}{r_2})ka \textbf(e)^{j(\omega t-kr_2)}|}\,\si{\decibel}
\end{equation}
\startexplain
\explain{$\Delta L_{p,a}$ is the sound pressure level difference.}{\si{\decibel}}
\explain{$r_1, r_2$ are the distances, for which the pressures are being calculated, \(r>a\).}{\si{\meter}}
\explain{$t$ is the time, for which the pressure is being calculated.}{\si{\meter}}
\explain{$\rho_0$ is the specific density of air.}{\si{\kilo\gram\per\cubic\meter}}
\explain{$c$ is the speed of sound.}{\si{\meter\per\second}}
\explain{$V_0$ ist the velocity at the surface of the spherical source.}{\si{\meter\per\second}}
\explain{$a$ is the radius of the spherical source.}{\si{\meter}}
%\explain{$\theta_a$ is equal to $\tan(ka)$.}{\si{1}}
\explain{$\omega$ is the angular frequency.}{\si{\second^{-1}}}
\explain{$k$ is the wavenumber.}{\si{\meter^{-1}}}
\stopexplain
\autoref{eq:deltalplong} can be simplified to a great extent when adapting into the context of the \gls{fdtd}-simulation:
\begin{equation}\label{eq:deltalpshort}
\Delta L_{p,a}(r_1,r_2)\,=\,20\log_{10}\frac{r_2}{r_1}\,\si{\decibel}
\end{equation}
When a single sound source is positioned in the \gls{fdtd}-grid, the \gls{rms}-pressure at two points with known distances to the source can be recorded. If level difference at these points approximately matches the value according to \autoref{eq:deltalpshort}, that can be interpreted as a sign for the simulation to work correctly.\\
%In terms of implementation a test like this is excecuted using the \texttt{unittest} module.
%MORE TEXT HERE!
In order to assess computation time, different implementations are run on the same machine with the same parameters and the time is measured. 
%In this context, the ``bottleneck'' of the computation can be determined and discussed: MORE TEXT HERE?
\subsection{Evaluation}\label{ssec:evaluation}
In order to get an rough estimate over which order of magnitude of precision can be achieved in \gls{fdtd}-simulation, a comparison test was set up. A single transparent sound source was set up amidst a three-dimensional cubic room with an edge length $D$. The grid size was set to $\delta d=\delta x=\delta y=\delta z = \SI{0.05}{\meter}$. The source radiated sound at a frequency of $f=\SI{300}{\hertz}$. The \gls{rms} pressure was logged at distances of $r_1=\SI{1}{\meter}$ and $r_2=\SI{2}{\meter}$ from the source. The difference between the simulated pressures $\Delta L_{p,s}=L_{r2}-L_{r1}$ was compared to the analytical pressure difference according to \autoref{eq:deltalpshort}. For the distances as stated above $\Delta L_{p,a}=\SI{6.021}{\decibel}$. The error will denoted as $\epsilon_{s}=\Delta L_{p,a} - \Delta L_{p,s}$. This means, that negative values of $\epsilon_{s}$ indicate, that the simulated pressure drop is bigger then it is expected according to the analytical model. Simulations were carried out with single- and double precision floats, in order to see if this impedes the error $\epsilon_s$.\\
\begin{table}[h]
\centering
\caption{Error evaluation}
\label{tab:pressure_difference}
\begin{tabular}{l|l|l}
      & \multicolumn{2}{c}{\(\epsilon_s\)\,[\si{\decibel}]} \\ \hline
\(D\)\,[\si{\meter}] & Single Precision & Double Precision \\ \hline
10    & - 0.686          & - 0.686          \\ \hline
20    & - 0.750          & - 0.750          \\ \hline
30    & - 0.040          & - 0.040         
\end{tabular}
\end{table} 

\autoref{tab:pressure_difference} indicates, that there is no advantage in using double precision variables over single precision variables. In an acoustical context, the results are virtually indistinguishable. Because single precision calculations offer increased performance, they are preferable. The simulation grid size appears to have an influence on the error $\epsilon_s$. Grid size and error do not appear to be reciprocal, but it can be noted, that $\epsilon_s$ is significantly smaller for a edge length of $D=\SI{30}{\meter}$ then at the other lengths investigated. Supposedly this is also depending on the grid resolution $\delta d$.\\
While the difference in error between single precision and double precision calculations can be deemed negligible, the error values for $D=\SI{10}{\meter}$ and $D=\SI{20}{\meter}$ might actually be relevant in some applications. More information on error in \gls{fdtd} simulation can e.g. be found in \citep{fdtderror}.



\begin{table}[h]
\centering
\caption{Velocity step calculation times in a 3D-grid, calculated on a computer with an I7 920 processor and \SI{12}{\giga\byte} RAM, (*) extrapolation from the time measured in a 2D-grid}
\label{tab:times}
\begin{tabular}{l|c|r|r|r|r|r}
                                         & \multicolumn{6}{c}{Calculation time [\si{\second}]}                                                                                                                             \\ \cline{2-7} 
                                         & \multicolumn{3}{c|}{Single Precision}                                                    & \multicolumn{3}{c}{Double Precision}                                                 \\ \hline
\multicolumn{1}{l|}{(D)\,[\si{\meter}]} & naive*                        & \multicolumn{1}{c|}{single} & \multicolumn{1}{c|}{multi} & \multicolumn{1}{c|}{naive*} & \multicolumn{1}{c|}{single} & \multicolumn{1}{c}{multi} \\ \hline
10                                       & \multicolumn{1}{r|}{200.00}   & 0.15                        & 0.19                       & 0.16                        & 0.37                        & 0.27                      \\
20                                       & \multicolumn{1}{r|}{2340.00}  & 2.19                        & 1.10                       & 27,37                       & 29.93                       & 20.02                     \\
30                                       & \multicolumn{1}{r|}{48150.00} & 53.38                       & 61.03                      & 81870.00                            & 251.8                            & 487.26                         
\end{tabular}
\end{table}

\autoref{tab:times} shows the times it took to calculate a single velocity step in a grid with the given edge lenght \(D\). In every scenario, except the \SI{10}{\meter} Double Precision grid, the naive implementation proves to be the slowest by far. The implementations that utilize array operations do not show an unambiguous behaviour. For some cases the multicore implementation has performance benefits. However, this seems to only be the case for grids within a certain sweet spot size, where the additional thread management begins to pay off and the computation is not yet slowed down by limitations related to the shared memory.
There is a clear tendency, that calculations in single precision save a significant amount of calculation time.\\
When observing the CPU workload during calculations, the values seem stay quite low with non-naive implementation. It can therefore be assumed, that calculations are memory bound.

In \autoref{fig:p_rms}, graphical results of the simulations, on which \autoref{tab:pressure_difference} is based, are shown. The simulation has been stopped at a point in time, at which the wave emitted from the sound source in the middle of the grid has just reached the boundary at the cardinal directions. No reflection has happened (yet). This corresponds to acoustical free field conditions. The plots have been truncated to only show a quadratic area, in which the pressure has already traveled. It has to be noted, that the simulations have been run in 3D, but the plots only show a two-dimensional heat map of the \gls{rms}-sound pressure level. The planes of the grid, that have been the source of this data, are at the same height (\texttt{z}-dimension) as the sound source.\\
Matching the findings from \autoref{tab:pressure_difference}, there is no visible difference between single precision and double precision based results. The sound fields in the different grid sizes look similar in quality. It is clearly visible, that pressure is radiated in a circular pattern from the omnidirectional source in the middle of each grid.
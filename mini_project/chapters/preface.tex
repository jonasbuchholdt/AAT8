\iflanguage{english}{
\chapter*{Preface\markboth{Preface}{Preface}}\label{ch:preface}%
\addcontentsline{toc}{chapter}{Preface}%
}{%
\chapter*{Forord\markboth{Forord}{Forord}}\label{ch:forord}%
\addcontentsline{toc}{chapter}{Forord}%
}
This Mini Project is composed by group 18gr872 during the 8th semester of \projectFaculty{} at \AAU{}. The general purpose of the report is the development and implementation of an \gls{fdtd} simulation for sound fields in Python and documenting aspects of scientific computing encountered while doing so. 

For citations, the report employs the Harvard method. If citations are not present by figures or tables, these have been made by the authors of the report. Units are indicated according to the SI standard.

%The base of numerical representations are denoted by subscript. If no base is specified the number is base 10.

%This project uses the Assembly language for the TMS320C5515 processor, and furthermore, the C programming standard C99.

\vspace{\baselineskip}\hfill \AAU, \today
\vfill\noindent
\begin{center}
\begin{minipage}[b]{0.45\textwidth}
 \centering
  \textit{}\\
 {}
\end{minipage}
\hspace{0.3cm}
\begin{minipage}[b]{0.45\textwidth}
 \centering
  \textit{}\\
 {}
\end{minipage}
\end{center}
\vspace{1\baselineskip}
\begin{center}
\begin{minipage}[b]{0.45\textwidth}
 \centering
  \textit{Jonas Buchholdt}\\
 {\footnotesize <jbuchh13@student.aau.dk>}
\end{minipage}
\hspace{0.3cm}
\begin{minipage}[b]{0.45\textwidth}
 \centering
  \textit{Christoph Kirsch}\\
 {\footnotesize <ckirsc17@student.aau.dk>}
\end{minipage}
\end{center}


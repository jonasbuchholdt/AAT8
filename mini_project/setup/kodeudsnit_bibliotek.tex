\definecolor{javared}{rgb}{0.6,0,0} % for strings
\definecolor{javagreen}{rgb}{0.25,0.5,0.35} % comments
\definecolor{javapurple}{rgb}{0.5,0,0.35} % keywords
\definecolor{javadocblue}{rgb}{0.25,0.35,0.75} % javadoc
\definecolor{gray}{rgb}{0.4,0.4,0.4}
\definecolor{darkblue}{rgb}{0.0,0.0,0.6}
\definecolor{cyan}{rgb}{0.0,0.6,0.6}
\definecolor{lightblue}{rgb}{0.0,0.3,0.7}
\definecolor{orange}{rgb}{0.8,0.3,0.0}
\definecolor{matlabstring}{rgb}{.627,.126,.941}


%Hvordan XML kode skal se ud
\lstdefinestyle{customXML}{
  belowcaptionskip=1\baselineskip,
  breaklines=true,
  frame=L,
  xleftmargin=\parindent,
  language=XML,
  showstringspaces=false,
  basicstyle=\footnotesize\ttfamily,
  morestring=[b]",
  morestring=[s]{>}{<},
  morecomment=[s]{<?}{?>},
  stringstyle=\color{black},
  identifierstyle=\color{darkblue},
  keywordstyle=\color{cyan},
  morekeywords={xmlns,version,type},
 commentstyle=\color{gray}\upshape,
}


%Hvordan java kode skal se ud
\lstdefinestyle{customjava}{
  belowcaptionskip=1\baselineskip,
  breaklines=true,
  frame=L,
  xleftmargin=\parindent,
  language=java,
  showstringspaces=false,
  basicstyle=\footnotesize\ttfamily,
  keywordstyle=\color{javapurple}\bfseries,
stringstyle=\color{javared},
commentstyle=\color{javagreen},
morecomment=[s][\color{javadocblue}]{/**}{*/},
}

%Hvordan C kode skal se ud
\lstdefinestyle{customc}{
  belowcaptionskip=1\baselineskip,
  breaklines=true,
  frame=L,
  xleftmargin=\parindent,
  language=C,
  showstringspaces=false,
  basicstyle=\footnotesize\ttfamily,
  keywordstyle=\bfseries\color{green!40!black},
  commentstyle=\itshape\color{gray},
  identifierstyle=\color{blue},
  stringstyle=\color{orange},
}

%Hvordan ASM kode skal se ud
\lstdefinestyle{customassembly}{
  belowcaptionskip=1\baselineskip,
  breaklines=true,
  frame=L,
  xleftmargin=\parindent,
  %language=assembly,
  showstringspaces=false,
  basicstyle=\footnotesize\ttfamily,
  keywordstyle=\bfseries\color{green!40!black},
%where there is a space  
  numbers=left,%
  numberstyle={\tiny \color{black}},% size of the numbers
  numbersep=9pt, % this defines how far the numbers are from the text
  commentstyle=\itshape\color{javagreen},
  %assembly structure
  emph=[1]{equ,text,data,word,set},emphstyle=[1]\color{matlabstring}, 
  %commands
  emph=[2]{MOV,mov,BTST,BCC,NOP,RET},emphstyle=[2]\color{gray}, 
  %label
  emph=[3]{in,recieve_flag,out,transmit_flag,buffer,positive,negative,final,case_one,case_two},emphstyle=[3]\color{blue},
  identifierstyle=\color{black},
  stringstyle=\color{blue},
    comment=[l];,                              % comments
}

%Hvordan Matlab kode skal se ud
\lstdefinestyle{custommatlab}{
basicstyle=\footnotesize\ttfamily,
    breaklines=true,%
    morekeywords={matlab2tikz},
    keywordstyle=\color{blue},
    %morekeywords=[2]{1}, 
    keywordstyle=[2]{\color{black}},
    identifierstyle=\color{black},
    morestring=[m]', % defines that strings are enclosed in double quotes
    stringstyle=\color{matlabstring},
    showstringspaces=false,%without this there will be a symbol in the places where there is a space
    numbers=left,%
    numberstyle={\tiny \color{black}},% size of the numbers
    numbersep=9pt, % this defines how far the numbers are from the text
    emph=[1]{if,for,end,break,while,else,function},emphstyle=[1]\color{blue}, %some words to emphasise
    emph=[2]{all,on},emphstyle=[2]\color{matlabstring}, 
% matlat languate definition
  comment=[l]\%,                              % comments
  morecomment=[l]...,                         % comments
  morecomment=[s]{\%\{}{\%\}},                % block comments
  commentstyle=\color{Green},
}

%Hvordan VHDL kode skal se ud
\lstdefinestyle{customVHDL}{
  belowcaptionskip=1\baselineskip,
  breaklines=true,
  frame=L,
  xleftmargin=\parindent,
  language=VHDL,
  showstringspaces=false,
  basicstyle=\footnotesize\ttfamily,
  keywordstyle=\bfseries\color{blue!100!black!80},
  commentstyle=\itshape\color{green!90!black!90},
  identifierstyle=\color{black},
  stringstyle=\color{orange},
}

%Hvordan PHP kode skal se ud
\lstdefinestyle{customPHP}{
  belowcaptionskip=1\baselineskip,
  breaklines=true,
  frame=L,
  xleftmargin=\parindent,
  language=PHP,
  showstringspaces=false,
  basicstyle=\footnotesize\ttfamily,
 keywordstyle    = \color{blue},
  stringstyle     = \color{gray},
  identifierstyle = \color{lightblue},
  commentstyle    = \color{green},
  emph            =[1]{php},
  emphstyle       =[1]\color{black},
}
%Commando til at indsætte kode i latex
\newcommand{\includeCode}[7]{\lstinputlisting[caption=#5 | #1, style=custom#2, numbers=left, firstnumber=#3, firstline=#3, lastline=#4, label=#6]{#7#1}}

%Caption navn foran nummeret i kode
\renewcommand{\lstlistingname}{Code snippet}
\def\lstlistingautorefname{Code snippet}